% В этом файле следует писать текст работы, разбивая его на
% разделы (section), подразделы (subsection) и, если нужно,
% главы (chapter).

% Предварительно следует указать необходимую информацию
% в файле SETUP.tex

\input{preamble.tex}

\begin{document}

\Intro

Здесь вам нужно писать введение. Некоторые часто используемые команды приведены в качестве примера ниже (и варианты — в комментариях).

Your introduction goes here! Some examples of commonly used commands and features are listed below, to help you get started. If you have a question, please use the help menu (``?'') on the top bar to search for help or ask us a question.

% Если typeOfWork в SETUP.tex задан как 2 или 3, то начинать
% надо не с section (раздел), а с главы (chapter)
\chapter{Синтаксис и семантика языка программирования}
\section{Синтаксис}
Рассмотрим язык программиования 
\section{Семантика}

\chapter{Основные термины}
\section{TMS отношение}

\chapter{Результаты}

\section{Доказательство теоремы 1 (Необходимость)}

\section{Доказательство теоремы 1 (Достаточность)}


\Conc

We hope you find \LaTeX\ useful, and please let us know if you have any feedback.

Помните, что на все пункты списка литературы должны быть ссылки. \LaTeX\ просто не добавит информацию об издании из bib"/файла, если на это издание нет ссылки в тексте. Часто студенты используют в работе  электронные ресурсы: в этом нет ничего зазорного при одном условии: при каждом заимствовании следует ставить соответствующую ссылку. В качестве примера приведём ссылку на сайт нашего института~\autocite{mmcs}.

Для дальнейшего изучения \LaTeX\ рекомендуем книгу Львовского~\autocite{Lvo2003}: она хорошо написана, хотя и несколько устарела.
Обычно стоит искать подсказки на
\href{http://tex.stackexchange.com/}{tex.stackexchange.com}, а также
читать документацию по установленным пакетам с помощью
команды
\begin{Verb}
texdoc имя_пакета
\end{Verb}
или на \href{http://ctan.org/}{ctan.org}.

% Печать списка литературы (библиографии)
\printbibliography[heading=bibintoc%
    %,title=Библиография % если хочется это слово
]
% Файл со списком литературы: biblio.bib
% Подробно по оформлению библиографии:
% см. документацию к пакету biblatex-gost
% http://ctan.mirrorcatalogs.com/macros/latex/exptl/biblatex-contrib/biblatex-gost/doc/biblatex-gost.pdf
% и огромное количество примеров там же:
% http://mirror.macomnet.net/pub/CTAN/macros/latex/contrib/biblatex-contrib/biblatex-gost/doc/biblatex-gost-examples.pdf

\end{document}
